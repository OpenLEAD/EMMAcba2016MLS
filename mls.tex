\section{Moving Least Squares for surface interpolation}
Moving Least Squares (MLS), introduced by Backus-Gilbert, is a method for
near-best approximations to functionals on $\mathbb{R}^d$, using scattered-data
information. The method works very well for interpolation, smoothing and
derivatives' approximations, and the local error is bounded in terms of the
error of a local best polynomial approximation \cite{levin1998approximation}.

In the specific case of surface interpolation, let $M := {n+m \choose n}$ be the
dimension of the space polynomials of degree at most $m$ in $n$ variables, and
$\{\phi_1(\textbf{x}),\phi_2(\textbf{x}),\cdots, \phi_M(\textbf{x})\}$ be the
monomials of degree at most $m$ in $n$ real variables. For instance, for
$m=2$ and $n=2$, $\Phi=\{1,x,y,x^2,y^2,xy\}$
